% !TeX program = lualatex

\documentclass[a4paper]{article}

% Packages %%%%%%%%%%%%%%%%%%%%%%%%%%%%%%%%%%%%%%%
\usepackage{relsize}
\usepackage{geometry}
\usepackage{titlepic}
%\usepackage[utf8]{inputenc} % inputenc is not required if lualatex is used
\usepackage{authblk}
\usepackage{fancyhdr}
\usepackage[urlcolor = accent_light,linktocpage]{hyperref}
\usepackage{tocloft}
\usepackage{lipsum}
\usepackage{graphicx}
\usepackage{float}
\usepackage{pdfpages}
\usepackage{comment}
\usepackage[normalem]{ulem}
\usepackage[none]{hyphenat}
\usepackage{sectsty}
\usepackage{listings}
\usepackage{tabulary}
\usepackage[backend=bibtex, style=ieee, citestyle=ieee]{biblatex}
\usepackage{subfiles}
\usepackage{fontspec}
\usepackage[nopostdot, acronym, toc]{glossaries}
\usepackage[nottoc, notlot, notlof]{tocbibind}
\usepackage{parskip}
\usepackage{enumitem}
\usepackage{svg}
\usepackage{csquotes}
\usepackage{hyperref}
\usepackage[acronym]{glossaries-extra}


% Overall settings %%%%%%%%%%%%%%%%%%%%%%%%%%%%%%%
\graphicspath{{images/}}
\svgpath{{images/svg/}}
\pagestyle{fancy}
\fancyhf{}


\definecolor{content_dark}   {RGB}{ 24, 26, 30}
\definecolor{content_normal} {RGB}{ 47, 51, 60}
\definecolor{content_light}  {RGB}{ 71, 76, 90}
\definecolor{content_bright} {RGB}{ 94,101,119}

\definecolor{accent_dark}    {RGB}{ 31, 40,  7}
\definecolor{accent_normal}  {RGB}{ 25, 80, 13}
\definecolor{accent_light}   {RGB}{ 46,146, 24}
\definecolor{accent_bright}  {RGB}{ 67,211, 34}

\definecolor{stress_dark}    {RGB}{186,123,  0}
\definecolor{stress_normal}  {RGB}{247,164,  0}
\definecolor{stress_light}   {RGB}{249,187, 64}
\definecolor{stress_bright}  {RGB}{251,210,128}

% text used to highlight something
\newcommand{\main}[1]{\textcolor{accent_normal}{\MakeUppercase{\textbf{#1}}}}

% text used to emphasis
\newcommand{\heavy}[1]{\textcolor{content_dark}{\large{\textbf{#1}}}}

% text used to briefly describe something
\newcommand{\brief}[1]{\textcolor{accent_light}{\textbf{#1}}}
	
% text used to comment/note something
\newcommand{\note}[1]{\textcolor{accent_light}{\textit{#1}}}

% legacy text which will be removed later
\newcommand{\legacy}[1]{\textcolor{content_bright}{\textbf{#1}}}

% very important text to be highlighted
\newcommand{\crucial}[1]{\colorbox{accent_bright}{\textcolor{content_dark}{\textbf{#1}}}}



\defaultfontfeatures{Mapping=tex-text,Scale=MatchLowercase}
\setmainfont{Asap}[
	SizeFeatures={Size=12}
]

\setmonofont{Nimbus Mono}


\hypersetup{
	colorlinks=true,
	linkcolor=accent_light,
	citecolor=accent_light,
}

\sectionfont{\color{accent_light}}
\subsectionfont{\color{accent_normal}}
\subsubsectionfont{\color{accent_dark}}

% Appendices %%%%%%%%%%%%%%%%%%%%%%%%%%%%%%%%%%%%%
\makeglossaries
%\setacronymstyle{sc-short-long}  
\loadglsentries{appendices/glossaries}

\addbibresource{appendices/references.bib}

% Glossary links repetition %%%%%%%%%%%%%%%%%%%%%%
\GlsXtrEnableLinkCounting[section]{general,acronym}
% disable hyperlink if link count is greater than 1:
\renewcommand*{\glslinkpresetkeys}{%
	\ifnum\GlsXtrLinkCounterValue{\glslabel}>1
	\setkeys{glslink}{hyper=false}%
	\fi
}

% Code listing style %%%%%%%%%%%%%%%%%%%%%%%%%%%%%
\lstdefinestyle{sharpc}{language=[Sharp]C, frame=lr, rulecolor=\color{accent_light!80!black}}

\definecolor{bluekeywords}{rgb}{0.13,0.13,1}
\definecolor{greencomments}{rgb}{0,0.5,0}
\definecolor{redstrings}{rgb}{0.9,0,0}

\lstset{language=[Sharp]C,
	showspaces=false,
	showtabs=false,
	breaklines=true,
	showstringspaces=false,
	breakatwhitespace=true,
	escapeinside={(*@}{@*)},
	commentstyle=\color{accent_light},
	keywordstyle=\color{stress_dark},
	stringstyle=\color{content_light}
}

%% Document settings %%%%%%%%%%%%%%%%%%%%%%%%%%%%%
% content settings
\newcommand{\project}{Projekt Hexxle}
\newcommand{\outline}{Projekt Hexxle\\Projektdokumentation}


\title{
	\Huge{}\color{accent_normal}\includesvg[width=\linewidth]{hexxle_logo_1}\\ 
	\vspace{2cm}
	\large{}\color{accent_dark}\textbf{\outline}
}


\author{
	Deniz Akca
	\and
	Dennis Bannerman
	\and
	Adriana de Sao José Martins
	\and
	Peter Michailis
	\and
	Ricardo Monteiro Simoes
	\and
	Andrew Surber
}
\affil{ZHAW - Zurich}


\rhead{\color{content_light}\includesvg[width=2cm]{hexxle_logo_2}}
\lhead{\color{content_light}\project\ $\vert$ \textit{\leftmark}}

\rfoot{\color{content_light}-\thepage-}


\renewcommand{\cftsecleader}{\cftdotfill{\cftdotsep}}


% Define titlepage layout
\makeatletter
\def\@maketitle{%
	\newpage
	\null
	\vskip 1cm%
	\begin{center}%
		\let \footnote \thanks
		{\LARGE \@title \par}%
		\vskip 2cm%
		{\large
			\lineskip .25em%
			\begin{tabular}[t]{c}%
				\@author
			\end{tabular}\par}%
		\vfill%
		{\large \@date}
	\end{center}%
	\par
	\vskip 1.5em
}
\makeatother

% \todo{Draft watermark, should be removed for final}
%\usepackage{draftwatermark}
%\SetWatermarkText{Draft}
%\SetWatermarkColor[gray]{0.9}
%\SetWatermarkAngle{60}
%\SetWatermarkScale{20}
% 

% Improvements for package todonotes, remove for final
\usepackage{xargs}                      % Use more than one optional parameter in a new commands
\usepackage[colorinlistoftodos,prependcaption,textsize=tiny]{todonotes}
\newcommandx{\unsure}[2][1=]{\todo[linecolor=red,backgroundcolor=red!25,bordercolor=red,#1]{#2}}
\newcommandx{\change}[2][1=]{\todo[linecolor=blue,backgroundcolor=blue!25,bordercolor=blue,#1]{#2}}
\newcommandx{\info}[2][1=]{\todo[linecolor=green,backgroundcolor=green!25,bordercolor=green,#1]{#2}}
\newcommandx{\improvement}[2][1=]{\todo[linecolor=violet,backgroundcolor=violet!25,bordercolor=violet,#1]{#2}}
\newcommandx{\thiswillnotshow}[2][1=]{\todo[disable,#1]{#2}}
%

% Document structure %%%%%%%%%%%%%%%%%%%%%%%%%%%%%
\begin{document}
	\sloppy
	\pagenumbering{roman}
	\color{content_normal}
	
	\begin{titlepage}
		\maketitle
		\thispagestyle{empty}
	\end{titlepage}

	\tableofcontents
	\newpage
	
	\pagenumbering{arabic}

	%\subfile{sections/example}
	
	\lstset{style=sharpc}
	
	%% PROJEKT VISION TEIL %%
	%\subfile{sections/productvision/produktidee}
	%\subfile{sections/productvision/spielregeln}
	%\subfile{sections/productvision/umsetzung}
	%\subfile{sections/productvision/architektur}
	%\subfile{sections/productvision/technologien}	
	
	%% PROJEKT DOKU TEIL %%
	\subfile{sections/productvision/introduction}
	\subfile{sections/context}
	\subfile{sections/objectives_functional_overview}
	\subfile{sections/quality_attributes_non_functional_requirements}
	\subfile{sections/constraints}
	\subfile{sections/principles}
	\subfile{sections/architecture} %incomplete
	\subfile{sections/external_interfaces}
	\subfile{sections/code}
	\subfile{sections/data} %ausbaufähig
	\subfile{sections/infrastructure_architecture} %currently EMPTY
	\subfile{sections/deployment} %currently EMPTY
	\subfile{sections/operation_support} 
	\subfile{sections/decision_log}	
	
	%% PROJEKT ZUSATZ %%
	\appendix
	\subfile{sections/sprint_reviews/sprint_0}
	\subfile{sections/sprint_reviews/sprint_1}
	\subfile{sections/sprint_reviews/sprint_2}
	\subfile{sections/sprint_reviews/sprint_3}
	\subfile{sections/sprint_reviews/sprint_4}
	\subfile{sections/appendix/spielregeln}
	\subfile{sections/productvision/risiken}
	\subfile{sections/productvision/miscellaneous}
	% add sections here
	
	\newpage

	\printglossary[type=\acronymtype]
	\printglossary
	
	\newpage
	\printbibliography[heading=bibintoc]
	
	\newpage
	\listoftodos[Notes]
\end{document}

% Die nachfolgenden Grundsätze sollten für die agile Dokumentation gelten:
%	•"Keep it short, keep it simple
%	"Niemand hat die Zeit umfangreiche Dokumentationen zu lesen. Noch weniger Spass macht es sie zu schreiben.
% 	Das Guidebook soll die notwendigen Informationen in kurzer und präziser Form enthalten.
%
%	•"As much, as currently needed" bzw. "Just enough upfront design"
%	Das Guidebook soll die Information enthalten, die Momentan benötigt wird. Einige Angaben sollten bereits zu
%	Beginn des Projektes geklärt werden (Qualitätsattribute, Beschränkungen, Prinzipien, etc.),
%	andere werden grob gestartet und während des Projektablaufs ständige ergänzt und verfeinert (Funktionale Übersicht,
%	Code, Daten, Entscheidungs Protokoll, ...) und weitere werden erst im Laufe des Projektes hinzugefügt, sobald Sie
%	benötigt werden (Deployment, Operation und Support, ...)
%
%	•"Allways up to date"
%	Das Guidebook ist ein lebendes Dokument, dass laufend erweitert und aktualisiert wird. Damit das funktioniert,
%	sollte es online bereitgestellt und leicht zugänglich sein. Wie jedes Dokument sollte es versioniert sein
%	(z.B. Wiki, Webseite, Github-Pages, etc). Word-Dokumente sind deslhalb denkbar schlecht geeignet
%
%
% Die Projektdokumentation (Software-Guidebook) startet in der Projekt-Setup-Phase als Projekt-Vision und soll während dem Projekt laufend ergänzt
% und nachgeführt werden. Die kontinuierliche Aktualisierung fliesst in jedem Sprint in die 2P Bewertung "Sprint-Durchführung" ein.
% Die Bewertung des Inhalts erfolgt nach Abschluss des fünften Sprints als Teil des Resultats und wird mit 0..10 Punkten bewertet.

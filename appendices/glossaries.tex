% dual entry command stolen from
% https://en.wikibooks.org/wiki/LaTeX/Glossary#Dual_entries_with_reference_to_a_glossary_entry_from_an_acronym
%
% used like:
% \newdualentry{label} % label
% {LBL}                % abbreviation
% {Label}              % long form
% {What's a label???}  % description
%
% Refer to acronym with \gls{OWD} and the glossary with \gls{gls-OWD}


\usepackage{xparse}

\DeclareDocumentCommand{\newdualentry}{ O{} O{} m m m m } {
	\newglossaryentry{gls-#3}{name={#5},text={#5\glsadd{#3}},
		description={#6},#1
	}
	\makeglossaries
	\newacronym[see={[Glossary:]{gls-#3}},#2]{#3}{#4}{#5\glsadd{gls-#3}}
}




% Acronym

\newacronym{acronymexample}
	{AE}
	{\textbf{A}cronym \textbf{E}xample - additional description}



% ------------------------------------------------

% Acronyms/Glossary mixed

\newdualentry{mvp}
	{MVP}
	{\textbf{M}inimal \textbf{V}iable \textbf{P}roduct}
	{
		Ein Minimum Viable Product ist das bezogen auf den Umfang kleinstmögliche, auslieferbare und vor allem funktionsfähige Produkt.
	}

% ------------------------------------------------

% Glossary 

\newglossaryentry{hexxle}
{
	name=Hexxle,
	description={
		Arbeitstitel des im Projekt entwickelten Spiels, Übernahme des Teams
	}
}

\newglossaryentry{windows}
{
	name=Windows,
	description={
		Ein von der Firma Microsoft entwickeltes Betriebssystem.
	}
}

\newglossaryentry{unity}
{
	name=Unity,
	description={
		Unity ist eine Entwicklungsumgebung, die sich auf das Entwickeln von digitalen Spielen spezialisiert.
	}
}

\newglossaryentry{tetris}
{
	name=Tetris,
	description={
		Ein im Jahre 1984 erschienenes Kultvideospiel, bei dem die Spieler*innen so lange wie möglich Blöcke aufeinander stapeln.
	}
}

\newglossaryentry{islanders}
{
	name=Islanders,
	description={
		\cite{IslandersWiki}: Ein im Jahre 2019 erschienenes Videospiel. Islanders setzt auf Gelegenheitsspieler, dreht sich um Städtebau und wurde vom deutschen Indie-Game-Studio Grizzly Games entwickelt.
	}
}

\newglossaryentry{dorfromantik}
{
	name=Dorfromantik,
	description={
		\cite{DorfromantikHomepage}: Ein sich in der Entwicklung befindendes Videospiel, das auf Puzzle-Aspekte und strategische Elemente setzt. Dorfronamtik wird durch das Studio Toukana Interactive entwickelt. (Stand 2021)
	}
}

\newglossaryentry{peerreview}
{
	name=Peer-Review,
	description={
		Peer-Review ist ein Verfahren zur Qualitätssicherung, bei dem ein Erzeugnis durch unabhängige bzw. unbeteiligte Gutachter auf seine ausreichende Qualität überprüft wird.
	}
}

\newglossaryentry{highscore}
{
	name=Highscore,
	description={
		Die höchste, bisher erreichte Punktzahl wird mit dem englischen Begriff Highscore bezeichnet.
	}
}

\newglossaryentry{csharp}
{
	name=C\#,
	description={
		C\# (gesprochen: \enquote{C Sharp}) ist eine Programmiersprache und wird benötigt um mit \gls{unity} zu arbeiten.
	}
}

\newglossaryentry{bestpractice}
{
	name=Best Practice,
	description={
		Best Practice oder auch Erfolgsmodell, -methode, -rezept nennt man Methoden oder Praktiken, die sich in der Vergangenheit bewährt und als optimal etabliert haben.
	},
	plural={Best Practices}
}

\newglossaryentry{feature}
{
	name=Feature,
	description={
		Der Begriff Feature steht allgemein für eine Eigenschaft oder ein Merkmal. In der Informatikindustrie wird Feature allerdings in der Regel verwendet, um eine besondere Funktion des Produkts zu beschreiben.
	},
	plural={Features}
}

\newglossaryentry{jira}
{
	name=JIRA,
	description={
		JIRA ist ein Produkt der Firma Atlassian, das Teams in unterschiedlichen Bereichen des Projektmanagements unterstützen soll.
	}
}

\newglossaryentry{itch.io}
{
	name=itch.io,
	description={
		\href{https://itch.io/}{Itch.io} ist eine Webseite zum Anbieten, Verkaufen und Herunterladen von Computerspielen und Software.
	}
}

\newglossaryentry{innosetupcompiler}
{
	name=Inno Setup Compiler,
	description={
		Inno Setup ist ein Tool zur Verpackung eines Programms in einen Installer. Dieser Installer kann anschliessend zur Installation des Programms verwendet werden.
	}
}

\newglossaryentry{userstories}
{
	name=User-Stories,
	description={
		Als User-Stories werden Software-Anforderungen bezeichnet, die in der Form einer Geschichte mit Subjekt und Objekt geschrieben ist. Die Entwickler können davon zu implementierende Tasks ableiten.
	}
}

\newglossaryentry{cicd}
{
	name=CI/CD,
	description={
		CI/CD (Continuous Integration / Continuous Delivery) beschreibt eine Praktik in der Software-Entwicklung, die dafür sorgt, dass die zu entwickelnde Software konstant in ein ausführbares Programm umgewandelt und somit von allen Stakeholdern getestet werden kann.
	}
}

\newglossaryentry{versionierung}
{
	name=Versionierung,
	description={
		Mittels Versionierung wird in der Software-Entwicklung eine reibungslose Kooperation der Entwickler ermöglicht. Ausserdem kann so eine Chronologie aller gemachten Veränderungen eingesehen werden.
	}
}

\newglossaryentry{testingpipeline}
{
	name=Testing-Pipeline,
	description={
		Mittels Github sind Pipelines möglich, die nach einer Änderung am Code ausgeführt werden. Eine dieser Pipelines ist dafür zuständig, alle erstellten Tests auszuführen und über deren Ergebnis zu berichten.
	}
}

\newglossaryentry{storypoints}
{
	name=Story-Points,
	description={
		\gls{userstories} werden in Tasks aufgeteilt und der Aufwand jener Tasks wird in Story-Points geschätzt. Story-Points müssen hierbei nicht zwingend einen realen Gegenwert haben, sondern sollen einfach nur die relative Komplexität wiederspiegeln.
	}
}

\newglossaryentry{fibonacciskala}
{
	name=Fibonacci-Skala,
	description={
		Die Fibonacci-Skala lehnt sich an die Fibonacci-Folge an, um \gls{storypoints} zu schätzen. Für \gls{hexxle} verwenden wir die werte 1,2,3,5 und 8.
	}
}

\newglossaryentry{score}
{
	name=Score,
	description={
		Damit ist ein Punktewert oder Punktestand gemeint.
	}
}

\newglossaryentry{sprint}
{
	name=Sprint,
	description={
		Die Dauer einer Iteration in SCRUM wird Sprint genannt. Ein Sprint umfasst Planning, Durchführung, Review und Retrospektive.
	},
	plural={Sprints}
}

\newglossaryentry{velocity}
{
	name=Velocity,
	description={
		Mit Velocity bezeichnet man die Rate, mit der das SCRUM-Team pro Sprint \gls{storypoints} umsetzt.
	}
}

\newglossaryentry{scrum}
{
	name=SCRUM,
	description={
		Eine Methodologie zur agilen Umsetzung von Projekten.
	}
}

% ------------------------------------------------

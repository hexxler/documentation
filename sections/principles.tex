\documentclass[../main.tex]{subfiles}

\begin{document}
    % Dieses Kapitel beinhaltet eine Liste von Softwareentwickungsprinzipien, die in diesem Projekt angewendet werden.
    % Hier sollen die wichtigen Prinzipien nochmals explizit und allen Teilneh-mern bewusst gemacht werden.
    % Wenn es bestehende übergreifende Policies gibt, kann man darauf verweisen.
    % Typische Beispielefür SWE-Prinzipien sind:
    %   •Verwendung von SOLID (Single responsiblity principle, Open/Closed principle, Liskov substi-tutions principle, Interface segregation principle, Dependency Inversion principle)
    %   •DRY (don't repeat yourself)
    %   •Hohe Kohäsion und minimale Koppelung
    %   •Verwendung von Schichtmodellen
    %   •Keine Business Logik im Views
    %   •kein Datenbankzugriffaus Views, Verwendung von ORM
    %   •Hollywood-Prinzip (don't call us, we call you)
    %   •Alle Komponenten sindstateless,
    %   •Ansatz für Fehlerbehandlung,
    %   •etc.
    %
    % Dieses Kapitel muss in jedem Software-Guidebook mehr oder weniger ausführlich enthalten sein.
    \section{Prinzipien}
    \label{section:Prinzipien}
    \par Das Team von \gls{hexxle} hat sich darauf geeinigt, einige Prinzipien festzulegen, unter welchen die Entwicklung stattfinden wird.
    
    \subsection{Clean Code Massnahmen}
    \label{section:CleanCodeMassnahmen}
    \par Die Entwickler halten sich an gängige Clean Code Standards. Darunter sind folgende Punkte besonders zu beachten:
    \begin{itemize}
    	\item DRY (Don't repeat yourself)
    	\item Single-Responsibility-Prinzip
    	\item Liskovsches Substitutionsprinzip
    	\item YAGNI (You ain't gonna need it)
    \end{itemize}
    \par Sollten diese in der Software-Branche gängigen Begriffe den Teammitgliedern nicht bekannt sein, sind sie verpflichtet, sich selbständig darüber zu informieren.
    
    \subsection{Coding Convention C\#}
    \label{section:CodingConvention}
    \par Die Teammitglieder befolgen die für \gls{csharp} gängige Coding Convention. Mehr dazu findet man auf \href{https://github.com/ktaranov/naming-convention/blob/master/C\%23\%20Coding\%20Standards\%20and\%20Naming\%20Conventions.md}{GitHub} und bei \href{https://docs.microsoft.com/en-us/dotnet/csharp/programming-guide/inside-a-program/coding-conventions}{Microsoft}.
    
	\subsection{Definition of Done}
	\label{section:DefinitionOfDone}
	\par Bevor eine Story als abgeschlossen gilt, muss gewährleistet sein, dass die folgenden Punkte erfüllt wurden:
	\begin{itemize}
		\item Die Story wurde entsprechend der Akzeptanzkriterien vollständige umgesetzt (d.h. auch alle verknüpften Tasks sind abgeschlossen). 
		\item Der neu hinzugefügte Codeteil wurde einem Peer-Review unterzogen und für genügend befunden.
		\item Bisherige End-to-End-Tests und Unit-Tests laufen erfolgreich ab bzw. wurden entsprechend der neuen Funktionalität angepasst.
		\item Nicht für sich selbst sprechender Code wurde hinreichend dokumentiert.
		\item Für den neu implementierten Code existieren Unit-Tests.
		\item \textbf{optional:} Für den neu implementierten Code existieren End-to-End-Tests.
		\item Manuelle Tests durch die Entwickler ergaben keine Fehler. 
		\item Das Feature bzw. die Story wurde durch den Product Owner als vollständig abgesegnet.
	\end{itemize}
\end{document}
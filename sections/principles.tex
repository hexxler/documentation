\documentclass[../main.tex]{subfiles}

\begin{document}
    % Dieses Kapitel beinhaltet eine Liste von Softwareentwickungsprinzipien, die in diesem Projekt angewendet werden.
    % Hier sollen die wichtigen Prinzipien nochmals explizit und allen Teilneh-mern bewusst gemacht werden.
    % Wenn es bestehende übergreifende Policies gibt, kann man darauf verweisen.
    % Typische Beispielefür SWE-Prinzipien sind:
    %   •Verwendung von SOLID (Single responsiblity principle, Open/Closed principle, Liskov substi-tutions principle, Interface segregation principle, Dependency Inversion principle)
    %   •DRY (don't repeat yourself)
    %   •Hohe Kohäsion und minimale Koppelung
    %   •Verwendung von Schichtmodellen
    %   •Keine Business Logik im Views
    %   •kein Datenbankzugriffaus Views, Verwendung von ORM
    %   •Hollywood-Prinzip (don't call us, we call you)
    %   •Alle Komponenten sindstateless,
    %   •Ansatz für Fehlerbehandlung,
    %   •etc.
    %
    % Dieses Kapitel muss in jedem Software-Guidebook mehr oder weniger ausführlich enthalten sein.

\end{document}
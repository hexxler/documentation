\documentclass[../main.tex]{subfiles}

\begin{document}
    % Schnittstelle, insbesondere externe, sind oft die risikoreichsten Elemente einer Software.
    % Es ist deshalb oft sinnvoll Zusammenzufassen, welche Interfaces wie verwendet werden.
    % Folgende Fragen sollte diese Sektion beantworten:
    %   •Was sind die externenSchlüssel-Schnittstellen?
    %       o Externe Schnittstellen, die Sie konsumieren
    %       o APIs die Sie für externe Systeme anbieten
    %       o Dateien, die sie exportieren/importieren
    %   •Wurde die Schnittstelle aus technischer Sicht durchdacht?
    %       o Technische Definition
    %       o Kommunikationsparadigma (REST, RPC, Pub/Sub)
    %       o Nachrichtenformat
    %       o Synchron/Asynchron
    %       o Sind asynchrone Nachrichten garantiert?
    %       o Ist das Interface Idempotent?
    %   •Wurde die Schnittstelle aus nicht technischer Sicht durchdacht?
    %       o Wer ist zuständig (ownership)
    %       o Wie werden Änderungen am Interface gehandhabt?
    %       o Sind SLAs vorhanden?
    % Dieses Kapitel muss enthalten sein, sofern externe Schnittstellen konsumiert oder angeboten werden


	\section{Externe Schnittstellen und Abhängigkeiten}
	\par Das Projekt wird als \gls{unity}-Projekt ohne jegliche Netzwerkkommunikation oder ähnliches umgesetzt. Somit existieren keinerlei Abhängigkeiten gegenüber externen Schnittstellen, die nicht im Zusammenhang zu Unity stehen. 
	\par Desweiteren beschränken sich die Unity spezifischen Abhängigkeiten auf die eigentliche Engine. Dies gilt natürlich nur solange, wie das Team darauf verzichtet, allfällige bestehnde Assets aus dem Unity-Asset-Store zu verwenden.
	\par Etwaige Unklarheiten bezüglich Unity lassen sich im Allgemeinen mittels der  \href{https://docs.unity3d.com/2019.4/Documentation/ScriptReference/index.html}{Onlinedokumentation} klären.
\end{document}
\documentclass[../main.tex]{subfiles}

\begin{document}
    % In diesem Teil wird das Mapping zwischen Software (e.g. Containern aus Software-Architektur Sektion) und der Infrastruktur beschrieben.
    % Folgende Fragen soll hier beantwortet werden:
    %   •Wie und wo wird die Software installiert und konfiguriert?
    %   •Was ist die Deployment / Rollback Strategie?
    %   •Ist das Deployment automatisiert?
    %   •Sind die Anleitungen für die Installation vorhanden und aktuell?
    %   •Welche Optionen müssen konfiguriert werden und was ist deren Bedeutung?
    % Varianten:
    %   •Tabelle welche das Mapping zwischen Software-Container und Infrastruktur zeigt
    %   •Deployment-Diagramm (modifiziertes Diagramm aus der Infrastruktur-Architektur, das auf-zeigt, wo welche Komponente installiert wird)
    % Hier sollte alle Information oder Referenzen vorhanden sein die dem Operations-Team Installation und Upgrade des Systems ermöglichen.
    % Die Installations/Deployment-Sektion sollte in jedem Fall im Guidebook enthalten sein.
	
	\section{Deployment}
	Das \gls{hexxle} Spiel wird vom Kunden individuell heruntergeladen. Dafür wird dem Kunden eine Installations-EXE zur Verfügung gestellt.
	Diese EXE wird mit dem Tool \gls{innosetupcompiler} erstellt und aus dem \gls{unity} Build des \gls{hexxle} Spiels erstellt.
	Der Installer stellt dem Kunden eine einfache Bedienoberfläche dar um die gängingsten Einstellungen wie Desktop-Shortcut und Installationsort zu definieren.
	Das \gls{hexxle} Spiel wird auf \gls{itch.io} veröffentlich und für den Erwerb erhältlich sein.
	Da das \gls{hexxle} Spiel ein einzelspieler Spiel wird kein Back-End-Server und dessen Deployment benötigt.

\end{document}
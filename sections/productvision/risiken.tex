\documentclass[../main.tex]{subfiles}

\begin{document}
	\section{Risikoanalyse}
	\subsection{Risiken}
	\par Obwohl das Projekt durch die diversifizierte Wissensbasis der einzelnen Teammitglieder in guten Händen ist, sind Risiken allgegenwärtig und müssen in jede Planung einfliessen. Dies erlaubt es uns, Vorgehen zu bestimmen, die das erfüllen des Projektziels in der verfügbaren Zeit sicherstellen. 
	
	\subsubsection{Zeitmanagement}
	\par Die Eckdaten zur Entwicklung und Implementation jeweiliger Features sind klar definiert, eben so wie das Fertigstellungsdatums. Fehlerhaftes Zeitmanagement kann entsprechend das ganze Projekt in Gefahr bringen. Dies wird wahrscheinlicher durch die mangelnde Erfahrung mit der Entwicklungsumgebung, die das Projekt voraussetzt. Dieser Umstand muss bei der Planung der einzelnen Sprints\improvement{glossar sprint} berücksichtigt werden.
	
	\subsubsection{Personenausfall}
	\par Zu unterscheiden gibt es temporäre Ausfälle (Krankheit o.Ä.) sowie indefinite Ausfälle (Verlassen des Teams). Da ein Ausfall einer Personen einen temporären bzw. konstanten Rückgang der Velocity\improvement{glossar velocity} mit sich bringen würde, ist das Gefahrenausmass gross. Im Umfeld dieser Projektarbeit ist der indefinite Ausfall eines Teammitglieds allerdings unwahrscheinlich.
	
	\subsubsection{Technische Probleme und Datenverlust}
	\par Das Auftreten technischer Probleme oder Datenverluste ist bei einem vollständig digitalen Projekt eine grosse Gefahr. Bereits implementierte Funktionalitäten zu verlieren oder Vergleichbares würde das Projekt stark verzögern. Durch korrektes Management mit Versionierungen\improvement{versionierung}, Cloud-Based-Solutions\improvement{cloud based solutions} o.Ä. lässt sich aber dieses Risiko gut eindämmen.
	
	\subsubsection{Entwicklung nicht benötigter Funktionalitäten}
	\par Durch die kurz bemessene Entwicklungszeit muss das Team sich strikt an die definierten Ziele halten. Das Entwickeln von nicht benötigter Funktionalität, ohne das hierfür Zeit bzw. explizit der Bedarf definiert wurde, kann das Zeitmanagement sowie den zeitigen Projektabschluss in Gefahr bringen.
	
	\subsubsection{Fehlende finanzielle Mittel}
	\par Da dies eine Projektarbeit unter Studierenden ist und darüber hinaus kein Profit vom Endprodukt erwartet werden kann, sind die finanziellen Mittel begrenzt. Dies kann zur Folge haben, dass vermehrt Funktionalitäten oder Assets team-intern entwickelt oder produziert werden müssen, was wiederum zu einem Mehraufwand führen kann.
	
	\subsubsection{Auswertung}
	\par In folgender Tabelle~\ref{tab:risks} wurden die Risiken entsprechend ihrer Wahrscheinlichkeit und ihrer Auswirkung auf das Projekt gewichtet und entsprechend bewertet:
	\begin{table}[H]
		\renewcommand*{\arraystretch}{1.05}
		\label{tab:risks}
		\centering
		\begin{tabular}{lrrr}
			Faktor                 & Wahrscheinlichkeit & Schadensausmass & Risikoauswertung \\
			Techn. Probleme        & 5                  & 4               & 20               \\
			Zeitmanagement         & 5                  & 3               & 15               \\
			Personenausfall        & 2                  & 5               & 10               \\
			Finanzen               & 3                  & 3               & 9                \\
			Redundante Entwicklung & 1                  & 2               & 2               
		\end{tabular}
		\caption{Auswertung der Risikoanalyse}
	\end{table}

	\subsection{Massnahmen}
	\par Bei jedem Risiko muss vorgängig definiert werden, wie das Team sich verhalten soll, um den Schaden zu minimieren, gesetzt den Fall, dass das entsprechende Ereignis eintritt.
	
	\subsubsection{Zeitmanagement}
	\par Der regelmässige Austausch im Team soll es ermöglichen, Arbeiten besser aufzuteilen sowie Unterstützung anbieten zu können, wo diese gebraucht wird. Durch diese Massnahme sollen Probleme bzgl. Zeitmanagement gelöst oder gar verhindert werden. Mittels dem Projektmanagement-Tool "Jira"\improvement{glossar jira} in Kombination mit SCRUM \improvement{glossar scrum} kann der Aufwand vorgängig geschätzt und eine Übersicht über aktuelle und zukünftige Pendenzen erstellt werden.
	
	\subsubsection{Personenausfall}
	\par Auch hier gilt es, einen konstanten Informationsaustausch im Team zu haben. Dadurch können bei einem Ausfall eines Teammitglieds seine Pendenzen auf das restliche Team aufgeteilt und schwerwiegende Auswirkungen auf den weiteren Verlauf des Projekts verhindert werden.
	
	\subsubsection{Technische Probleme und Datenverlust}
	\par Durch das Nutzen von Git für die Versionierung, sowie GitHub für das Cloud-Basierte Speichern jeglicher Dateien wird verhindert, dass Daten verloren gehen können. Die Verantwortung liegt allerdings bei den einzelnen Teammitgliedern, darauf zu achten, dass entwickelte Funktionalitäten stets synchronisiert werden. 
	\par Der Ausfall bzw. ein technisches Problem eines Endgerätes käme einem Personenausfall gleich und wird dementsprechend gehandhabt.
	
	\subsubsection{Entwicklung nicht benötigter Funktionalitäten}
	\par Mittels Jira und SCRUM \improvement{glossar} wird für jeden Sprint ein Zeitbudget definiert und dieses auf verschiedene Tasks aufgeteilt. Diese sollen dann entsprechend von den Teammitgliedern abgearbeitet werden.  Der Product Owner ist hier in der Verantwortung, dass falsche Funktionalitäten früh genug erkannt werden und so kein Zeitverlust entsteht.
	
	\subsubsection{Fehlende finanzielle Mittel}
	\par Kostenpflichtige Verbesserungen, Assets und Funktionalitäten werden so gut wie möglich durch kostenfreie Alternativen ersetzt. Wo dies nicht möglich ist, muss Zeit für den entstehenden Mehraufwand eingeplant werden.
\end{document}
\documentclass[../main.tex]{subfiles}

\begin{document}
	\section{Projekt Hexxle}
	\subsection{Produktvision}
	\label{section:Produktvision}
	\todo[inline]{Worauf zielen wir mit unserer Entwicklung ab? Was soll umgesetzt werden? Was ist geplant?}
	\par Das Ziel dieses Projekts ist die Entwicklung eines Videospiels. Dieses Videospiel bietet einen Einzelspieler-Modus und kann als Puzzlespiel klassifiziert werden. Die Regeln für diese Puzzlespiel sollen sich auf eine überschaubare Menge begrenzen. Die Entwicklung findet unter anderem mithilfe der \gls{unity} Entwicklungsumgebung statt.
	\par Am Ende des Projekts soll eine fertige, auf \gls{windows} spielbare Version des Spiels verfügbar sein. Der Umfang dieser Version hängt von den Entwicklungsarbeiten und deren Geschwindigkeit ab. Er soll jedoch mindestens die im Kapitel \nameref{section:DefinitionOfDone} aufgeführten Punkte erfüllen.
	\par Die genauen Spielregeln von \gls{hexxle} werden im Kapitel \nameref{section:Spielregeln} näher beschrieben.
	\subsection{Inspiration}
	\par Als Inspiration dienen die Spiele \gls{tetris}, \gls{islanders} und \gls{dorfromantik} sowie vereinzelte Mechaniken, die sich in diversen Brettspielen wiederfinden lassen. Die Kernmechanik von Islanders, Dorfromantik und im entfernten Sinne auch Tetris dreht sich um das geschickte Platzieren unterschiedlicher Objekte, so dass diese von anderen umliegenden Objekten profitieren und der Spieler somit seine Punkte maximieren kann.
\end{document}
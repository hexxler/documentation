\documentclass[../main.tex]{subfiles}

\begin{document}
	\section{Spielregeln}
	\label{section:Spielregeln}
	\todo[inline]{Wie funktioniert das Spiel?}
%Spielaufbau
	\subsection{Spielaufbau}
	\par Das Spiel besteht aus einem Spielfeld, einem Plättchenstapel, einer Plättchenauslage, einem Punktebalken bzw. einer Punkteanzeige und drei Buttons (`Zug rückgängig machen´, `Plättchenmarkt aufrufen´ und `Menu aufrufen´). 

%Ziel des Spiels
	\subsection{Ziel des Spiels}
	\par Das Ziel des Spiels ist es, eine möglichst hohe Punktzahl zu erreichen. Die Spieler*innen versuchen deshalb, die ihnen zur Verfügung stehenden Plättchen möglichst geschickt auf einer vorgegebenen Fläche anzuordnen. Jedes neu gesetzte Plättchen generiert eine Punktzahl basierend auf der Art der umliegenden Plättchen und des Plättchens selbst. Wird nach einigen Plättchen eine vorgegebene Punktzahl erreicht oder überschritten, erhalten die Spieler*innen weitere Plättchen. Sollten den Spieler*innen zu einem beliebigen Zeitpunkt keine Plättchen mehr zur Verfügung stehen, ist das Spiel zu Ende.
	
%Spielablauf
	\subsection{Spielablauf}
	\par Nachdem die Spieler*innen das Spiel gestartet haben, können sie eine neue Partie starten. Nähere Erklärungen zu einzelnen Begriffen finden sich im Kapitel \nameref{section:Spielmechaniken}.
	\subsubsection{Beginn einer Partie}
	\par Zu Beginn einer Partie stehen den Spieler*innen ein leeres Feld, eine volle Plättchenauslage und ein Plättchenstapel zur Verfügung. Die Plättchen innerhalb des Plättchenstapels und der Plättchenauslage sind zufällig. Der Punktestand der Spieler*innen ist zu diesem Zeitpunkt noch auf Null.
	\par Die Spieler*innen können sich nun für eines der Plättchen in ihrer Plättchenauslage entscheiden und dieses ohne Einschränkungen auf dem Spielfeld platzieren. Nachdem die Spieler*innen ihr erstes Plättchen platziert haben, erhalten sie eine entsprechende Anzahl Punkte. Der nun leere Platz in der Plättchenauslage wird mit dem obersten Plättchen des Plättchenstapels aufgefüllt.
	\subsubsection{Platzieren weiterer Plättchen}
	\par Ist das erste Plättchen platziert, gilt für das Platzieren weiterer Plättchen zusätzlich die Bedingung, dass sie jeweils mit mindestens einer Kante an ein bereits ausliegendes Plättchen angrenzen müssen. Auch für das Platzieren dieser Plättchen erhalten die Spieler*innen Punkte.
	\par Sollte nach dem Platzieren eines Plättchens ein Intervall des Punktestandes überschritten werden, so bietet sich den Spieler*innen die Möglichkeit, pro abgeschlossenem Intervall jeweils einmal neue Plättchen im Plättchenmarkt zu erwerben. Diese Möglichkeit verfällt erst nachdem die Spieler*innen tatsächlich Plättchen im Plättchenmarkt erworben haben. Neu erworbene Plättchen werden oben auf dem Plättchenstapel abgelegt.
	\par Haben die Spieler*innen weder in ihrem Plättchenstapel, noch in ihrer Plättchenauslage weiter Plättchen zur Verfügung, die sie auslegen können, so ist die Partie zu Ende.

%Spielmechaniken
	\subsection{Spielmechaniken}
	\label{section:Spielmechaniken}
	\subsubsection{Plättchen}
	\par Die Spielsteine, mit denen die Spieler*innen Punkte generieren können, werden Plättchen genannt. Plättchen haben die folgenden Eigenschaften:
	\begin{itemize}
		\item \textbf{Plättchentyp} Jedes Plättchen hat einen Typ, der definiert, wie das Plättchen mit anderen Plättchen des gleichen Typs oder anderer Typen bzgl. der generierten Punkte interagiert.
		\item \textbf{Plättchenart} Jedes Plättchen hat einen Typ, der definiert, unter welchen Umständen ein Plättchen andere Plättchen als benachbart betrachtet.
		\item \textbf{Plättchenverhalten} Jedes Plättchen hat ein Verhalten, das darüber entscheidet, wie benachbarte Teilchen behandelt werden. Möglich sind:
		\begin{itemize}
			\item Benachbarte Plättchen bleiben
			\item Benachbarte Plättchen ändern ihren Typ
			\item Benachbarte Plättchen werden konsumiert 
		\end{itemize}
	\end{itemize}

	\subsubsection{Plättchenauslage}
	\par Spieler*innen greifen auf eine Auslage an Plättchen zu. Plättchen aus dieser Auslage können im Spielfeld abgelegt werden, um Punkte zu generieren. Wird ein Plättchen abgelegt, wird der frei gewordenen Platz in der Ablage mit dem obersten Plättchen des Plättchenstapels gedeckt. Sollte die Plättchenauslage jemals leer sein und keine Möglichkeit mehr bestehen, neue Plättchen über den Plättchenmarkt zu erwerben, endet das Spiel.
	
	\subsubsection{Plättchenstapel}
	\par Spieler*innen verfügen über einen Stapel an Plättchen. Die Reihenfolge der Plättchen im Stapel ist zufällig und nicht beeinflussbar. Sollte der Plättchenstapel an irgendeinem Punkt im Spiel leer sein, ist es nicht länger möglich, frei gewordene Plätze in der Plättchenauslage zu decken. Spieler*innen können dem Plättchenstapel zusätzliche Plättchen hinzufügen, indem sie Punkte generieren und sich so neue Plättchen erspielen.
	
	\subsubsection{Plättchenmarkt}
	\par Haben die Spieler*innen genügend Punkte erreicht, können sie im Plättchenmarkt neue, zufällige Plättchensets auswählen. Plättchensets haben eine Tendenz zu einem gewissen Plättchentyp und/oder einer Plättchenart, können aber auch unabhängige Plättchen enthalten. Neu erhaltene Plättchen werden oben auf den Plättchenstapel abgelegt.
	
	\subsubsection{Punktestand}
	\par Alle erhaltenen Punkte werden dem Punktestand beigefügt. Der Punktestand ist in wachsende Intervalle unterteilt. Für jedes dieser Intervalle, welches überschritten wird, erhalten die Spieler*innen die Möglichkeit, neue Plättchen im Plättchenmarkt zu ergattern.
	\par Der Punktestand wird einerseits in Form einer absoluten Punkteanzeige und andererseits in seinen Intervallen auch als Punktebalken repräsentiert.
\end{document}
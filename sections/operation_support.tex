\documentclass[../main.tex]{subfiles}

\begin{document}
    % Nahezu alle Systeme müssen betrieben und gewartet werden.
    % Deshalb sollte auch eine dedizierte Sektion dazu im Guidebook enthalten sein die explizit beschreibt, wie die Systeme überwacht, betrieben und administriert werden müssen.
    % FolgendeFragen sollten in diesem Teil angesprochen werden:
    %   •Ist definiert, welche Überwachungs-und Support-Funktionen die Software unterstützt?
    %   •Wie wird das auf allen Schichten der Anwendung sichergestellt?
    %   •Wo werden Fehler und Statusinformationen gelogged?
    %   •Wie werden die Daten/Logs ausgewertet/aufbewahrt?
    %   •Gibt es manuelle Administrationsaufgaben?
    %   •Muss bei einer Konfigurationsänderung das System neu gestartet werden?
    %   •etc.
    % Der Operation und Support Teil sollte im Guidebook auf jeden Fall enthalten sein
    
    \section{Operation und Support}

    \subsection{Operation}
    \par Das Spiel \gls{hexxle} wird immer lokal auf einem Rechner laufen. Während der ganzen Spielzeit kommuniziert \gls{hexxle} mit keinen aussenstehenden Programmen. Auch läuft das Spiel niemals kontinuierlich ohne Unterbrechungen. Aus diesem Gründen wird auf einen Überwachungs-tool verzichtet. 
   
   \par Es liegt in der Verantwortung des Entwicklungsteams sicherzustellen, dass bei jedem Release eine lauffähige Version des Spiels zur Verfügung steht und dass möglichst wenig neue Bugs eingeführt werden. 

    \subsection{Support}
    \par Falls einem Spieler beim spielen des Spieles \gls{hexxle} ein unerwünschten Bug auffällt, kann der Spieler diesen dem Entwicklungsteam per Github (\url{https://github.com/hexxler/hexxle_game}) melden. Der Spieler kann auf der Github-Page einen neuen Issue erstellen mit detaillierten Reproduktionsschritte. Das Entwicklungsteam meldet sich danach nach maximal einer Woche. Es wird danach überprüft ob es sich dabei um einen Bug handelt. Falls dies der Fall ist, wird dieser im Backlog als Bug erfasst und beim nächsten Sprint gefixt. 
 

\end{document}

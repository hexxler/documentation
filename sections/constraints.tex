\documentclass[../main.tex]{subfiles}

\begin{document}
    % Software wird in der realen Welt eingesetzt, in welcher sie oft gewisse Beschränkungen ausge-setzt ist.
    % Diese sind typischerweise von aussen vorgegeben und definieren den Rahmen in wel-chem das Projekt sich bewegen kann.
    % Auch hier sollte eine einfache Auflistung reichen. Beispiele für Einschränkungen sind:
    %   •Zeit,
    %   •Budget und Ressourcen
    %   •Technologievorgaben
    %   •Ziel-Plattform
    %   •Bestehende Systeme und Integrationsstandards
    %   •lokale Standards
    %   •öffentliche Standards (HTTP, REST, WSDL, ...)
    %   •Standardprotokolle
    %   •Standard Nachrichtenformate
    %   •Grösse des Entwicklungsteams
    %   •Kompetenzprofil des Entwicklungsteams
    %   •taktisches oder strategische Produkte
    %   •politische Einschränkungen
    %   •etc.
    %
    %   Dieses Kapitel muss in jedem Software-Guidebook enthalten sein
	\section{Rahmenbedingungen}
	\par Das Team entwickelt \gls{hexxle} unter diversen Rahmenbedingungen, die vergangen, gegenwärtige und zukünftige Entscheide merkbar beinflussen. Im folgenden werden die wichtigsten Punkte aufgeführt.
	
	\subsection{Zeitliche Begrenzung}
	\par Da das Projekt im Rahmen einer Kursveranstaltung umgesetzt wird, ist es zeitlich auf die Dauer eines Schulsemesters beschränkt.
	
	\subsection{Ziel-Plattform}
	\par Auch wenn \gls{unity} dafür bekannt ist, eine breite Masse an Endplattformen zu bedienen, hat sich das Team dennoch entschieden, sich vorerst nur auf Geräte mit einem Windows-Betriebssystem zu fokussieren.
	
	\subsection{Technologievorgaben}
	\par Die Entwicklung eines Videospiels mithilfe von \gls{unity} bedeutet, dass das Team mit der Programmiersprache \gls{csharp} arbeiten wird.
\end{document}
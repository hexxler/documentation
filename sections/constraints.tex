\documentclass[../main.tex]{subfiles}

\begin{document}
    % Software wird in der realen Welt eingesetzt, in welcher sie oft gewisse Beschränkungen ausge-setzt ist.
    % Diese sind typischerweise von aussen vorgegeben und definieren den Rahmen in wel-chem das Projekt sich bewegen kann.
    % Auch hier sollte eine einfache Auflistung reichen. Beispiele für Einschränkungen sind:
    %   •Zeit,
    %   •Budget und Ressourcen
    %   •Technologievorgaben
    %   •Ziel-Plattform
    %   •Bestehende Systeme und Integrationsstandards
    %   •lokale Standards
    %   •öffentliche Standards (HTTP, REST, WSDL, ...)
    %   •Standardprotokolle
    %   •Standard Nachrichtenformate
    %   •Grösse des Entwicklungsteams
    %   •Kompetenzprofil des Entwicklungsteams
    %   •taktisches oder strategische Produkte
    %   •politische Einschränkungen
    %   •etc.
    %
    %   Dieses Kapitel muss in jedem Software-Guidebook enthalten sein

\end{document}
\documentclass[../main.tex]{subfiles}

\begin{document}
    % Schlussendlich sollten wichtige Entscheidungen betreffend Technologie (z.B. Produkte, Frame-works, etc.)
    % und Architektur (z.B. Softwarestruktur, Still, Patterns, etc.), die während der Entwicklung getroffen wurden, festgehalten werden.
    % Dies ermöglicht später im Betrieb oder bei der Weiterentwicklung nachzuvollziehen warum diese Entscheidung getroffen wurde.
    % Beispiele für Inhalte:
    %   •Warum wurde Technologie / Framework X gegenüber Y vorgezogen?
    %   •Wie wurde die Entscheidung untermauert? (evaluation, proof of concept, performance tes-ting,...)
    %   •Warum wurde die betreffende Softwarearchitektur gewählt? Welche anderen Optionen wur-den geprüft?
    % Die Beschreibung soll kurz gehalten werden (1 Abschnitt pro Entscheidung)
    \section{Entscheidungsprotokoll}
    
    %% Technologie %%
    \subsection{Technologie}
    
    \subsubsection{SonarQube}
    \par Der Entscheid gegen SonarQube wird Anhand der gegebenen Situation Entschieden. Es ist im verfügbaren Zeitrahmen zur Erstellung eines MVP nicht möglich, einen eigenen SonarQube-Analyzer zu erstellen. Dies ist nötig, da die verfügbaren Optionen nicht für unsere Entwicklungsumgebung funktioniert bzw. verwertbare Resultate erzielt. Einstimmig hat das Team beschlossen, mittels rigorosen Code-Reviews den Code-Standard hoch zu halten. \todo[inline]{Warum haben wir uns gegen SonarQube entschieden?}
    
    %% Personelles %%
    \subsection{Personelles}
    \subsubsection{Abgang Peter Michailis}
    \par Kurz nach Ende von Sprint 0 hat Peter Michailis uns mitgeteilt, dass es ihm leider nicht länger möglich ist, an diesem Projekt teilzunehmen. Dies hatte private Gründe und wurde so vom restlichen Team akzeptiert. Nichtsdestotroz war Peter ein wichtiger Teil des Teams und seine Kompetenz floss massgeblich in die Planung des Projekts ein. 
    \par Glücklicherweise war die geplante Version des \gls{mvp} bereits so gewählt, dass auch die Umsetzung mit einem Team bestehend aus fünf Personen möglich sein sollte. Die Reduktion der Teamgrösse bedeutet allerdings, dass wir diejenigen Aufgaben, die für Peter vorgesehen waren, nun auf das restliche Team aufteilen werden und am Ende des Projekts so gegebenenfalls weniger Zeit für zusätzliche \glspl{feature} bleibt.
\end{document}
\documentclass[../main.tex]{subfiles}

\begin{document}
    % Während der grösste Teil des Guidebooks sich mit der Software selber beschäftigt,
    % wird hier die physikalische/virtuelle Infrastruktur in welche die Software installiert und betrieben wird beschrieben.
    % Folgende Fragen sollten hier behandelt werden:
    %   •Ist eine klare Infrastruktur Architektur vorhanden?
    %   •Welche Hardware (virtuell oder physikalisch) wird dazu verwendet?
    %   •Ist Redundanz, Failoverund Disaster-Recovery vorgesehen?
    %   •Ist die Skalierung der Infrastruktur bekannt?
    %   •Wer ist zuständig für Support und Unterhalt der Infrastruktur?
    %   •Wer ist verantwortlich für die Ressourcen (ownership)?
    %   •etc.
    % Sehr oft enthält dieses Kapitel ein Infrastruktur/Netzwerk-Diagramm, welchesdie verschiedenen Hardware/Softwarekomponenten beschreibt und aufzeigt wie diese zusammenhängen.
    % Dieses Kapitel sollte in jedem Software-Guidebook enthalten sein.

\end{document}
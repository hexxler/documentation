\documentclass[../main.tex]{subfiles}

\begin{document}
    % In den bisherigen Sektionen war der Fokus nicht primär auf die Daten gerichtet, obwohl diese in modernen Systemen eine essentielle Rolle spielen.
    % Ziel dieser Sektion ist die Informationen zu dokumentieren, die aus Datenperspektive wichtig sind. Auch hier soll jede Sektion kurz gehalten werden.
    % Eventuell ein Entity-Relationship Diagramm, wenn es dem Leser hilft.
    %   •Wie sieht das Datenmodell aus?
    %   •Wo sind die Daten gespeichert?
    %   •Wer ist verantwortlich für die Daten (ownership)?
    %   •Was ist das Mengengerüst? Wie viel welcher Daten fallen an?
    %   •Archivierung und Backup Strategien
    %
    % Dies ist ein optionales Kapitel, das nur verwendet wird, wenn notwendig. Bei Daten-getriebenen Systemen jedoch sehr häufig der Fall
	\section{Datenmodell}
	\par \gls{hexxle} ist auf Einzelspieler ausgelegt. Daten werden somit nicht ausserhalb der jeweiligen Maschine gelagert. 
	\par Zum jetzigen Zeitpunkt ist \gls{hexxle} auf die jeweilige Session beschränkt, weswegen keine Konzepte zur Speicherung von Daten wie etwa der aktuellen Spielrunde erstellt wurden.

\end{document}